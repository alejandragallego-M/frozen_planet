\documentclass{article}
\usepackage{graphicx}

\title{Frozen Planet} 
\author{Alejandra Marulanda Gallego}

\begin{document}
\maketitle Article 1-Potential energy surfaces in atomic structure: 
The role of Coulomb correlation in the ground state of 
helium.

\begin{abstract}

"For the S states of two-electron atoms, we introduce an exact
and unique factorization of the internal eigenfunction in terms
of a marginal amplitude, which depends functionally on the 
electron-nucleus distances r1 and r2, and a conditional 
amplitude, which depends functionally on the interelectronic 
distance r12 and parametrically on r1 and r2."
 - Donde podemos mirar los conceptos básicos?, como lo de 
 amplitud marginal... (Densidad de probabilidad)
 - La amplitud condicional la define uno? en que se basan
 para definir tales condicionales?     

"Applying the variational principle, we derive pseudoeigenvalue
equations for these two amplitudes, which cast the internal 
Schrodinger equation in a form akin to the Born-Oppenheimer 
separation of nuclear and electronic degrees of freedom in 
molecules."
 - Principio variacional? Método para no usar aproximaciones?
Principio variacional, no se puede demostrar es un principio. 

"The marginal equation involves an effective radial Hamiltonian,
which contains a nonadiabatic potential energy surface that takes 
into account all interparticle correlations in an averaged 
way, and whose unique eigenvalue is the internal energy."
 - Trabajar en superficie de energía no abiabática significa
 trabajar sin aproximaciones? como es una superficie no 
 adiabática? 
 - Que el valor propio solo sea la energía interna es consecuencia
 de trabajar con superficies no adiabáticas o por como definen la
 la ecuación marginal?

"At each point (r1,r2), such surface is, in turn, the 
unique eigenvalue in the conditional
equation. Employing the ground state of He as prototype, 
we show that the nonadiabatic potential energy surface 
affords a molecularlike interpretation of the structure 
of the atom, and aids in the analysis of energetic and 
spatial aspects of the Coulomb correlation, in particular 
correlation-induced symmetry breaking and quantum phase 
transition"
 - Con las superficias adiabáticas no se puede obtener la 
 interpretación tipo molecular de la estructura del átomo?

\end{abstract}
"In the quantum-mechanical framework, this notion is put 
in on the basis of the topography of a Born-Oppenheimer 
(BO) potential energy surface (PES)"
 - La noción de que las moléculas y otros agregados atómicos
 poseen formas bien definidas se puede basar en otro tipo de 
 superficies de energía portencial o solo está la de BO?

- Cómo es un sistema doblemente excitado? 
\begin{document}
\section*{Concepts}
\begin{itemize}
    \item Conditional
    \item Marginal function
\end{itemize}
\includegraphics[width=8cm, height=5cm, scale=0.5]{figura.jpg}
%\begin{figure}[htb]
%    \centering
%    \includegraphics[width=10cm, height=8cm, scale=0.5]{figura.jpg}
%    \caption{Caos cuántico.}
%\end{figure}
\end{document}